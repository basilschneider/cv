%%% Set internalNotes to true to also print the internal notes; default is false
\newif\ifinternalNotes
\internalNotestrue

%%% Spacing used between publication entries
\def\spacingPubs{-12pt}
%%% Width of parbox used to right align dates
\def\parboxWidthOne{19pt}
\def\parboxWidthTwo{16pt}

\documentclass[]{cv} % Add 'print' as an option into the square bracket to remove colors from this template for printing

%%% Set PDF metadata
\hypersetup{colorlinks, pdfpagemode=UseOutlines, pdftitle=cv, pdfauthor={B. Schneider}}

%\addbibresource{bibliography.bib} % Specify the bibliography file to include publications

\begin{document}

\header{Basil }{Schneider}{} % Your name and current job title/field

%----------------------------------------------------------------------------------------
%	SIDEBAR SECTION
%----------------------------------------------------------------------------------------

\begin{aside} % In the aside, each new line forces a line break
  \section{contact}
  Rheinstrasse 63
  4410 Liestal
  Switzerland
  ~
  +41 (0) 78 710 37 55
  +1-630-827-2826
  ~
  \href{mailto:basil.schneider@cern.ch}{basil.schneider@cern.ch}
  \section{languages}
  German (native)
  English (fluent)
  French (moderate)
  \section{computing}
  Linux
  C++, Python
  Root, RooFit, RooStats
  bash, sed, awk
  git, svn
  HTML, CSS
  \LaTeX
  \section{besides physics}
  Cycling
  Hiking
  Music
\end{aside}

%----------------------------------------------------------------------------------------
%	EDUCATION SECTION
%----------------------------------------------------------------------------------------

\section{education \& employment}

\begin{entrylist}
%------------------------------------------------
  \entry
  {\parbox[t]{\parboxWidthOne}{Nov}\parbox[t]{\parboxWidthTwo}{\hfill '15} - now}
  {\textbf{Research Associate} at the \textbf{CMS experiment}}
  {FNAL}
  {}
%------------------------------------------------
  \entry
  {\parbox[t]{\parboxWidthOne}{Nov}\parbox[t]{\parboxWidthTwo}{\hfill '14} - Oct '15}
  {\textbf{Postdoctoral Fellow} at the \textbf{ATLAS experiment}}
  {TRIUMF}
  {}
%------------------------------------------------
  \entry
  {\parbox[t]{\parboxWidthOne}{Jan}\parbox[t]{\parboxWidthTwo}{\hfill '11} - Jul '14}
  {\textbf{Ph.D.} at the \textbf{ATLAS experiment}}
  {University of Bern}
  {Ph.D. Thesis: A general approach to search for supersymmetry at the LHC by combining signal enhanced kinematic regions using the ATLAS detector (Supervisor:
  Prof. A. Ereditato)}
%------------------------------------------------
  \entry
  {\parbox[t]{\parboxWidthOne}{Sep}\parbox[t]{\parboxWidthTwo}{\hfill '08} - Mar '10}
  {\textbf{Master} of Science in \textbf{Theoretical Physics}}
  {ETH Zurich}
  {Master Thesis: The partition function of meromorphic conformal field theories at higher genus (Supervisor: Prof. M. Gaberdiel)}
%Overall grade: 5.13 (6 is the highest, 1 is the lowest; passmark is 4)}
%------------------------------------------------
  \entry
  {\parbox[t]{\parboxWidthOne}{Oct}\parbox[t]{\parboxWidthTwo}{\hfill '04} - Sep '08}
  {\textbf{Bachelor} of Science in \textbf{Experimental Physics}}
  {ETH Zurich}
  {Bachelor Thesis: Untersuchung der Cluster-Struktur von Elastomerpartikeln durch Simulation des Aggregationsvorganges und Partikelgr{\"o}ssen mittels dynamic
light scattering (Supervisor: Dr. Cornelius Gauer)}
%Overall grade: 5.07}
\entry
{\parbox[t]{\parboxWidthOne}{Sep}\parbox[t]{\parboxWidthTwo}{\hfill '04}}
{\textbf{Comprehensive entrance exam}}
{ETH Zurich}
{Exam at the level of a Matura}
%------------------------------------------------
\end{entrylist}

\section{conferences}

\begin{entrylist}

  \entry
  {\parbox[t]{\parboxWidthOne}{Oct}\parbox[t]{\parboxWidthTwo}{\hfill '17}}
  {\textbf{IEEE Nuclear Science Symposium and Medical Imaging Conference}}
  {}
  {Poster: ``A new DAQ solution: OTSDAQ''}

  \entry
  {\parbox[t]{\parboxWidthOne}{Aug}\parbox[t]{\parboxWidthTwo}{\hfill '17}}
  {\textbf{Meeting of the Division of Particles and Fields of the American Physical Society}}
  {}
  {Speaker: ``Searches for electroweakly produced supersymmetry with CMS''}

  \entry
  {\parbox[t]{\parboxWidthOne}{May}\parbox[t]{\parboxWidthTwo}{\hfill '17}}
  {\textbf{Phenomenology 2017 Symposium}}
  {}
  {Speaker: ``Searches for supersymmetry in single or opposite-charged dilepton final states with CMS''}

  \entry
  {\parbox[t]{\parboxWidthOne}{Jun}\parbox[t]{\parboxWidthTwo}{\hfill '16}}
  {\textbf{49th Annual Fermilab Users Meeting}}
  {}
  {Poster: ``Characterization of the pixel ASIC with a laser beam in the Outer Tracker upgrade of the CMS detector''}

  \entry
  {\parbox[t]{\parboxWidthOne}{Jun}\parbox[t]{\parboxWidthTwo}{\hfill '15}}
  {\textbf{USATLAS Workshop at University of Illinois at Urbana-Champaign}}
  {}
  {Invited speaker for plenary session: ``Supersymmetry in Run-2''}

  \entry
  {\parbox[t]{\parboxWidthOne}{May}\parbox[t]{\parboxWidthTwo}{\hfill '15}}
  {\textbf{Mitchell Workshop on Collider and Dark Matter Physics}}
  {\href{}{}}
  {Speaker: ``Supersymmetry searches in ATLAS''}

  \entry
  {\parbox[t]{\parboxWidthOne}{May}\parbox[t]{\parboxWidthTwo}{\hfill '13}}
  {\textbf{1\textsuperscript{st} LHC Physics Conference, Barcelona, Spain}}
  %{\href{https://cds.cern.ch/record/1555743}{ATL-PHYS-SLIDE-2013-350}}
  {}
  {Poster: ``Search for direct production of charginos and neutralinos in events with three
    leptons and missing transverse momentum in 21 fb\textsuperscript{-1} of pp collisions at $\sqrt{\mathsf{s}}$ = 8 TeV with the ATLAS
  detector''}

  \entry
  {\parbox[t]{\parboxWidthOne}{Jun}\parbox[t]{\parboxWidthTwo}{\hfill '12}}
  {\textbf{Swiss Physical Society}}
  {}
  {Speaker: ``New Optical receiver modules for the insertable B-Layer at the ATLAS project''}

  \entry
  {\parbox[t]{\parboxWidthOne}{Jun}\parbox[t]{\parboxWidthTwo}{\hfill '11}}
  {\textbf{Physics at LHC, Perugia, Italy}}
  %{\href{https://cds.cern.ch/record/1371922}{ATL-PHYS-SLIDE-2011-423}}
  {}
  {Poster: ``SUSY Searches at ATLAS in Multilepton Final States with Jets and Missing Transverse Energy''}

  \entry
  {\parbox[t]{\parboxWidthOne}{Jun}\parbox[t]{\parboxWidthTwo}{\hfill '11}}
  {\textbf{Swiss Physical Society}}
  {}
  {Speaker: ``Insertable b-Layer: A new layer for the ATLAS detector at CERN''}

\end{entrylist}

\section{organization}

\begin{entrylist}

  \entry
  {\parbox[t]{\parboxWidthOne}{Aug}\parbox[t]{\parboxWidthTwo}{\hfill '12}}
  {\textbf{Co-organizer of workshop: SUSY Statistical Interpretations workshop}}
  {}
  {Wrap up lessons learned in previous round of publications and spot possible
  improvements for next round}

  \entry
  {\parbox[t]{\parboxWidthOne}{Sep}\parbox[t]{\parboxWidthTwo}{\hfill '11}}
  {\textbf{Co-organizer of outreach event: Nacht der Forschung}}
  {}
  {Performing experiments in public and discussing results}

\end{entrylist}

\newpage

\section{papers}
\begin{entrylist}

  \entry
  {}
  {I am co-author of 475 ATLAS publications and 99 CMS publications;\\
  for a full list, see \\
    \href{http://inspirehep.net/author/profile/B.Schneider.1}{http://inspirehep.net/author/profile/B.Schneider.1}\\
  Publications where my contributions are substantial:}
  {}
  {\vspace*{\spacingPubs}}

  \entry
  {\parbox[t]{\parboxWidthOne}{Sep}\parbox[t]{\parboxWidthTwo}{\hfill '15}}
  {Search for the electroweak production of supersymmetric particles in
    $\sqrt{\mathsf{s}}$~=~8 TeV pp collisions with the ATLAS detector} 
    {\href{http://arxiv.org/abs/1509.07152}{arXiv:1509.07152 [hep-ex]}}
  {\vspace*{\spacingPubs}}

  \entry
  {\parbox[t]{\parboxWidthOne}{May}\parbox[t]{\parboxWidthTwo}{\hfill '14}}
  {Search for supersymmetry in events with four or more leptons in $\sqrt{\mathsf{s}}$ = 8 TeV pp collisions with the ATLAS detector}
  {\href{http://link.aps.org/doi/10.1103/PhysRevD.90.052001}{Phys. Rev. D. 90, 052001 (2014)}}
  {\vspace*{\spacingPubs}}

  \entry
  {\parbox[t]{\parboxWidthOne}{Feb}\parbox[t]{\parboxWidthTwo}{\hfill '14}}
  {Search for direct production of charginos and neutralinos in events with three leptons and missing transverse momentum in $\sqrt{\mathsf{s}}$ =
8 TeV pp collisions with the ATLAS detector}
{\href{http://dx.doi.org/10.1007/JHEP04(2014)169}{JHEP04(2014)169}}
{\vspace*{\spacingPubs}}

    \entry
    {\parbox[t]{\parboxWidthOne}{Aug}\parbox[t]{\parboxWidthTwo}{\hfill '12}}
    {Search for direct production of charginos and neutralinos in events with three leptons and missing transverse momentum in $\sqrt{\mathsf{s}}$ =
  7 TeV pp collisions with the ATLAS detector}
  {\href{http://dx.doi.org/10.1016/j.physletb.2012.11.039}{Phys.Lett. B718 (2013) 841-859}}
  {\vspace*{\spacingPubs}}

\end{entrylist}

\section{public notes}
\begin{entrylist}

  \entry
  {\parbox[t]{\parboxWidthOne}{Jul}\parbox[t]{\parboxWidthTwo}{\hfill '15}}
  {First look at proton proton collision data at $\sqrt{\mathsf{s}}$ = 13 TeV in
  preparation for a search for squarks and gluinos in events with missing
  transverse energy, jets, and an isolated electron or muon}
  {\href{https://cds.cern.ch/record/2037906}{ATL-PHYS-PUB-2015-029}}
  {\vspace*{\spacingPubs}}

  \entry
  {\parbox[t]{\parboxWidthOne}{Mar}\parbox[t]{\parboxWidthTwo}{\hfill '15}}
  {Expected sensitivity studies for gluino and squark searches using the early
  LHC 13 TeV Run-2 dataset with the ATLAS experiment\\}
  {\href{https://cds.cern.ch/record/2002608/}{ATL-PHYS-PUB-2015-005}}
  {\vspace*{\spacingPubs}}

  \entry
  {\parbox[t]{\parboxWidthOne}{Jun}\parbox[t]{\parboxWidthTwo}{\hfill '14}}
  {A general approach to search for supersymmetry at the LHC by combining signal
  enhanced kinematic regions using the ATLAS detector (PhD thesis)\\}
  {\href{https://cds.cern.ch/record/1709233/}{CERN-THESIS-2014-056}}
  {\vspace*{\spacingPubs}}

  \entry
  {\parbox[t]{\parboxWidthOne}{Mar}\parbox[t]{\parboxWidthTwo}{\hfill '13}}
  {Search for supersymmetry in events with four or more leptons in 21 fb\textsuperscript{-1} of pp collisions at
  $\sqrt{\mathsf{s}}$ = 8 TeV with the ATLAS detector}
  {\href{https://cds.cern.ch/record/1532429}{ATLAS-CONF-2013-036}}
  {\vspace*{\spacingPubs}}

  \entry
  {\parbox[t]{\parboxWidthOne}{Mar}\parbox[t]{\parboxWidthTwo}{\hfill '13}}
  {Search for direct production of charginos and neutralinos in events with three leptons and missing transverse momentum in 21
    fb\textsuperscript{-1} of pp collisions at $\sqrt{\mathsf{s}}$ = 8 TeV with the ATLAS detector}
    {\href{https://cds.cern.ch/record/1532426}{ATLAS-CONF-2013-035}}
    {\vspace*{\spacingPubs}}

    \entry
    {\parbox[t]{\parboxWidthOne}{Nov}\parbox[t]{\parboxWidthTwo}{\hfill '12}}
    {Search for direct production of charginos and neutralinos in events with three leptons and missing transverse momentum in 13.0
      fb\textsuperscript{-1} of pp collisions at $\sqrt{\mathsf{s}}$ = 8 TeV with the ATLAS detector}
      {\href{https://cds.cern.ch/record/1493493}{ATLAS-CONF-2012-154}}
      {\vspace*{\spacingPubs}}

      \entry
      {\parbox[t]{\parboxWidthOne}{Nov}\parbox[t]{\parboxWidthTwo}{\hfill '12}}
      {Search for Supersymmetry in events with four or more leptons in 13 fb\textsuperscript{-1} pp collisions at $\sqrt{\mathsf{s}}$
    = 8 TeV with the ATLAS detector}
    {\href{https://cds.cern.ch/record/1493492}{ATLAS-CONF-2012-153}}
    {\vspace*{\spacingPubs}}

\end{entrylist}

\section{proceedings}
\begin{entrylist}

\entry
{\parbox[t]{\parboxWidthOne}{Jun}\parbox[t]{\parboxWidthTwo}{\hfill '13}}
{Search for direct production of charginos and neutralinos in events with three leptons and missing transverse momentum in 21
  fb\textsuperscript{-1} of pp collisions at $\sqrt{\mathsf{s}}$ = 8 TeV with the ATLAS detector}
  {\href{https://cds.cern.ch/record/1554811}{ATL-PHYS-PROC-2013-145}}
  {\vspace*{\spacingPubs}}

  \entry
  {\parbox[t]{\parboxWidthOne}{Nov}\parbox[t]{\parboxWidthTwo}{\hfill '11}}
  {The ATLAS IBL BOC Demonstrator}
  {\href{https://cds.cern.ch/record/1401224}{ATL-INDET-PROC-2011-038}}
  {\vspace*{\spacingPubs}}

  \entry
  {\parbox[t]{\parboxWidthOne}{Oct}\parbox[t]{\parboxWidthTwo}{\hfill '11}}
  {SUSY Searches at ATLAS in Multilepton Final States with Jets and Missing Transverse Energy}
  {\href{https://cds.cern.ch/record/1394331}{ATL-PHYS-PROC-2011-201}}
  {\vspace*{\spacingPubs}}

\end{entrylist}

\newpage
\ifinternalNotes
\section{internal notes}

\begin{entrylist}

  \entry
  {\parbox[t]{\parboxWidthOne}{Jul}\parbox[t]{\parboxWidthTwo}{\hfill '15}}
  {First look at proton proton collision data at $\sqrt{\mathsf{s}}$ = 13 TeV in
  preparation for a search for squarks and gluinos in events with missing
  transverse energy, jets, and an isolated electron or muon}
{\href{https://cds.cern.ch/record/2034389}{ATL-COM-PHYS-2015-718}}
{\vspace*{\spacingPubs}}

  \entry
  {\parbox[t]{\parboxWidthOne}{May}\parbox[t]{\parboxWidthTwo}{\hfill '15}}
  {ATLAS Large eta task force report}
{\href{https://cds.cern.ch/record/2016324}{ATL-COM-UPGRADE-2015-013}}
{\vspace*{\spacingPubs}}

  \entry
  {\parbox[t]{\parboxWidthOne}{Feb}\parbox[t]{\parboxWidthTwo}{\hfill '15}}
  {Expected sensitivity of search for squarks and gluinos in events with
    isolated leptons, jets and missing transverse momentum at
    $\sqrt{\mathsf{s}}$ = 13 TeV with the ATLAS detector}
{\href{https://cds.cern.ch/record/1994195}{ATL-COM-PHYS-2015-133}}
{\vspace*{\spacingPubs}}

  \entry
  {\parbox[t]{\parboxWidthOne}{Jan}\parbox[t]{\parboxWidthTwo}{\hfill '15}}
  {Re-interpretations and combinations of electroweak limits in
    20.3 fb\textsuperscript{-1} pp collisions at $\sqrt{\mathsf{s}}$
  = 8 TeV with the ATLAS detector}
{\href{https://cds.cern.ch/record/1981548}{ATL-COM-PHYS-2015-011}}
{\vspace*{\spacingPubs}}

  \entry
  {\parbox[t]{\parboxWidthOne}{Dec}\parbox[t]{\parboxWidthTwo}{\hfill '13}}
  {Search for supersymmetry in events with four or more leptons in $\sqrt{\mathsf{s}}$ = 8 TeV pp collisions with the ATLAS
detector}
{\href{https://cds.cern.ch/record/1635455}{ATL-COM-PHYS-2013-1621}}
{\vspace*{\spacingPubs}}

\entry
{\parbox[t]{\parboxWidthOne}{Oct}\parbox[t]{\parboxWidthTwo}{\hfill '13}}
{Search for supersymmetry in events with three leptons and missing transverse momentum in 21 fb\textsuperscript{-1} pp collisions
at $\sqrt{\mathsf{s}}$ = 8 TeV with the ATLAS detector}
{\href{https://cds.cern.ch/record/1610185}{ATL-PHYS-INT-2013-021}}
{\vspace*{\spacingPubs}}

\entry
{\parbox[t]{\parboxWidthOne}{Jul}\parbox[t]{\parboxWidthTwo}{\hfill '13}}
{Search for supersymmetry in events with three leptons and missing transverse momentum in 20.3 fb\textsuperscript{-1} pp
collisions at $\sqrt{\mathsf{s}}$ = 8 TeV with the ATLAS detector}
{\href{https://cds.cern.ch/record/1558985}{ATL-COM-PHYS-2013-888}}
{\vspace*{\spacingPubs}}

\entry
{\parbox[t]{\parboxWidthOne}{Dec}\parbox[t]{\parboxWidthTwo}{\hfill '12}}
{Search for Supersymmetry in events with four or more leptons in 20.7~fb\textsuperscript{-1} pp collisions at $\sqrt{\mathsf{s}}$
= 8 TeV with the ATLAS detector}
{\href{https://cds.cern.ch/record/1501709}{ATL-COM-PHYS-2012-1819}}
{\vspace*{\spacingPubs}}

\entry
{\parbox[t]{\parboxWidthOne}{Dec}\parbox[t]{\parboxWidthTwo}{\hfill '12}}
{Search for Supersymmetry in events with four or more leptons in 13 fb\textsuperscript{-1} pp collisions at $\sqrt{\mathsf{s}}$ =
8 TeV with the ATLAS detector}
{\href{https://cds.cern.ch/record/1498627}{ATL-PHYS-INT-2012-096}}
{\vspace*{\spacingPubs}}

\entry
{\parbox[t]{\parboxWidthOne}{Dec}\parbox[t]{\parboxWidthTwo}{\hfill '12}}
{Search for supersymmetry in events with three leptons and missing transverse momentum in 13 fb\textsuperscript{-1} pp collisions
at $\sqrt{\mathsf{s}}$ = 8 TeV with the ATLAS detector}
{\href{https://cds.cern.ch/record/1498390}{ATL-PHYS-INT-2012-095}}
{\vspace*{\spacingPubs}}

\entry
{\parbox[t]{\parboxWidthOne}{Sep}\parbox[t]{\parboxWidthTwo}{\hfill '12}}
{SUSY Searches in the Final States with Three Leptons and Missing Transverse Momentum at ATLAS}
{\href{https://cds.cern.ch/record/1482141}{ATL-PHYS-INT-2012-059}}
{\vspace*{\spacingPubs}}

\end{entrylist}
\fi

\section{supervision}

\begin{entrylist}

  \entry
  {\parbox[t]{\parboxWidthOne}{Dec}\parbox[t]{\parboxWidthTwo}{\hfill '14} -
Nov '15}
{\textbf{Felix Cormier}}
  {}
  {MSc student, University of British Columbia}

  \entry
  {\parbox[t]{\parboxWidthOne}{Nov}\parbox[t]{\parboxWidthTwo}{\hfill '14} -
Nov '15}
{\textbf{Matthew Gignac}}
  {}
  {PhD student, University of British Columbia}

  \entry
  {\parbox[t]{\parboxWidthOne}{Dec}\parbox[t]{\parboxWidthTwo}{\hfill '12} -
Mar' 14}
{\textbf{Benjamin Gerber}}
  {}
  {MSc student, University of Bern}

\end{entrylist}

\section{teaching}

\begin{entrylist}

  \entry
  {\parbox[t]{\parboxWidthOne}{Jan}\parbox[t]{\parboxWidthTwo}{\hfill '11} - May '14}
  {\textbf{Lab Course}}
  {University of Bern}
  {Supervising and assisting Physics undergraduate students working on fundamental experiments in mechanics and electronics}

  \entry
  {\parbox[t]{\parboxWidthOne}{Jan}\parbox[t]{\parboxWidthTwo}{\hfill '11} - May '14}
  {\textbf{Physics for Biologists}}
  {University of Bern}
  {Assisting 1\textsuperscript{st} year Physics course}

  \entry
  {\parbox[t]{\parboxWidthOne}{Jul}\parbox[t]{\parboxWidthTwo}{\hfill '11} - May '14}
  {\textbf{Private lessons for high-school graduates}}
  {Interlink Schulberatung GmbH}
  {Private lessons in Mathematics, Statistics and Physics}

  \entry
  {2007/2008}
  {\textbf{Teaching assistant}}
  {ETH Zurich}
  {Teaching assistant for environmental science students in Calculus}

\end{entrylist}

\newpage

%\section{experience}
%
%\begin{entrylist}
%
%  \entry
%  {\parbox[t]{\parboxWidthOne}{Jan}\parbox[t]{\parboxWidthTwo}{\hfill '11} - now}
%  {\textbf{Analysis: Supersymmetry}
%  \vspace{3pt}
%  \begin{itemize}
%    \item Binning and optimizing the three light lepton final state signal region
%    \item Signal region optimizations in three lepton final states with taus
%    \item Assembling and scripting a package of various scripts to facilitate limit computations; also used by other analyses
%    \item Statistical interpretations of results: Discovery p-values, exclusion contours, upper limits on model cross-sections
%    \item Simultaneous fit of the WZ background in control and signal regions
%    \item Optimization of distributed computing resources for limit calculations with pseudo-experiments
%    \item Presented the analyses at 2 SUSY-Approvals on behalf of the multilepton group
%    \item Assisting younger students
%  \end{itemize}}
%  {}
%  {\vspace*{-22pt}}
%
%  \entry
%  {\parbox[t]{\parboxWidthOne}{Jan}\parbox[t]{\parboxWidthTwo}{\hfill '11} - Dec '12}
%  {\textbf{Hardware: IBL}
%  \vspace{3pt}
%  \begin{itemize}
%    \item Building up the IBL activities at the University of Bern
%    \item Test of the optical readouts
%    \item Fesibility study of the usage of commercial products
%    \item Selecting best optical readout candidate based on experimental data
%    \item Rudimentary tests with FPGA programming in VHDL
%    \item Replacing dead optical transmitters in USA15
%    \item Presented studies at 2 IBL general meetings on behalf of the ATLAS Bern group
%  \end{itemize}}
%  {}
%  {\vspace*{-22pt}}
%
%\end{entrylist}

\section{outreach}

\begin{entrylist}

  \entry
  {\parbox[t]{\parboxWidthOne}{Nov}\parbox[t]{\parboxWidthTwo}{\hfill '13} - now}
  {\textbf{Official ATLAS underground guide}}
  {}
  {}

  \entry
  {\parbox[t]{\parboxWidthOne}{Mar}\parbox[t]{\parboxWidthTwo}{\hfill '12} - Mar
'13}
  {\textbf{Masterclasses}}
  {}
  {Helping high school students performing measurements on real data from LHC}

  \entry
  {\parbox[t]{\parboxWidthOne}{Sep}\parbox[t]{\parboxWidthTwo}{\hfill '11}}
  {\textbf{Nacht der Forschung}}
  {}
  {Presenting LHC physics on a poster and answering questions of the public in a
  research outreach event at the University of Bern}

\end{entrylist}

\section{awards}

\begin{entrylist}

  \entry
  {\parbox[t]{\parboxWidthOne}{Mar} '15}
  {\textbf{Faculty award winner of the University of Bern}}
  {}
  {Award for the best PhD thesis in physics at the University of Bern in the
  year 2014}

\end{entrylist}

\section{references}

\begin{entrylist}

  \entry
  {}
  {\textbf{Prof. Antonio Ereditato}}
  {PhD supervisor}
  {Laboratory for High Energy Physics, University of Bern\\+41 31 631 8566, \href{mailto:antonio.ereditato@cern.ch}{antonio.ereditato@cern.ch}}

  \entry
  {}
  {\textbf{Prof. Michele Weber}}
  {PhD supervisor}
  {Laboratory for High Energy Physics, University of Bern\\+41 31 631 5146, \href{mailto:weber@lhep.unibe.ch}{weber@lhep.unibe.ch}}

  \entry
  {}
  {\textbf{Dr. Jamie Boyd}}
  {SUSY convenor, external referee PhD thesis}
  {CERN\\+41 76 473 08 77, \href{mailto:jamie.boyd@cern.ch}{jamie.boyd@cern.ch}}

  \entry
  {}
  {\textbf{Dr. Anadi Canepa}}
  {SUSY Electroweak convenor}
  {TRIUMF, Canada's National Laboratory for Particle and Nuclear Physics\\+1 604-222-7330, \href{mailto:canepa@triumf.ca}{canepa@triumf.ca}}

  \entry
  {}
  {\textbf{Dr. Christina Potter}}
  {SUSY Electroweak convenor}
  {Department of Physics and Astronomy, University of Sussex\\+44 1273 873523, \href{mailto:christina.potter@sussex.ac.uk}{christina.potter@sussex.ac.uk}}

  \entry
  {}
  {\textbf{Dr. Tobias Flick}}
  {IBL Off-Detector Coordinator}
  {Detector Laboratory, Bergische Universit{\"a}t Wuppertal\\+49 202 439-2811, \href{mailto:flick@physik.uni-wuppertal.de}{flick@physik.uni-wuppertal.de}}

\end{entrylist}

\begin{statement}

\section{research statement: physics program}

Despite its success, the Standard Model of Particle Physics is known to be only
an effective theory. The most well known extension to the Standard Model is
Supersymmetry (SUSY). It could solve several short comings of the Standard
Model, most notably the hierarchy problem and the missing dark matter candidate.
The understanding of the nature of dark matter is one of the biggest goals of
physics right now. Since we discovered the Higgs Boson, SUSY searches will
become the driving force for finding thitherto unknown particles. It is of
outmost importance to carry on these efforts, even when first searches didn't
show any hint of SUSY particles. The unknown SUSY breaking mechanism could drive
the particles to higher masses without interfering with naturalness arguments.

Strongly charged SUSY particles have the highest production cross-section for a
given mass at the LHC. However, no signs of these particles have been found so
far and it might be that electroweak SUSY particles have a considerable lower
mass and become the dominant SUSY production process at the LHC. During my term
as a PhD student at the University of Bern I mainly worked on an electroweak
SUSY search in a final state with three leptons. I developed a method to
optimize and bin a signal region. By applying my method, I successfully pushed
existing exclusion limits on SUSY particle masses by about 100 GeV in a specific
model, without the use of additional data. I was the leading analyst in this
search and also carried out the statistical interpretation of the search
results. The published paper
(\href{http://dx.doi.org/10.1007/JHEP04(2014)169}{JHEP04(2014)169}) was also
presented in the CERN courier journal. In total I worked on three papers and
four ATLAS-CONF notes, all with a final state of either three or four leptons.
For two publications, I represented the analysis group on a SUSY approval. I'm
still working in close collaboration with the SUSY electroweak group to help
finalizing the Run-1 legacy paper that summarises all searches for
electroweakinos with the 8 TeV dataset.

When the LHC underwent a technical overhaul, the center of mass energy was
raised from 8~TeV to 13~TeV. With the new dataset, I contributed to searches for
strongly produced SUSY, since it has the highest discovery potential with small
amount of data. It was shown
(\href{https://cds.cern.ch/record/2002608/}{ATL-PHYS-PUB-2015-005}) that for a
final state with one lepton, even a few inverse femtobarn are enough, to have
better sensitivity as the full 8 TeV dataset. As example in a given model, a
gluino with mass 1.4~TeV could already show a 3~$\sigma$ excess with
5~fb$^{-1}$, where the existing 95~\% confidence level exclusion limit is
between 1.2 and 1.3~TeV.

For the upcoming searches, naturalness arguments can lead the way. Hence a
search for light stop's or gluino's should be carried out at high priority.
Searches for light electroweakinos will become relevant, once we have
accumulated a sufficiently large amount of data.

\section{detector}

As a PhD student I worked on the readout electronics of the insertable $b$-layer
(IBL). The IBL is a fourth layer which was inserted as innermost layer into the
existing three layer pixel detector. It was the main upgrade activity during the
first long shutdown in 2014 and 2015.

I tested optical receivers to be used for the readout for reliability, frequency
and input sensitivity. I defined the tests and performed them, using amongst
other tools FPGA's, which I programmed with VHDL. The University of Bern joined
the IBL effort and the contribution to the optical receivers testing made Bern a
key collaborator within the IBL readout community.

For my postdoctoral fellow position at TRIUMF, I joined the Inner Tracker (ITk)
effort, a new tracker to be implemented during the phase 2 upgrade for the High
Luminosity LHC around 2025. I started by using the validation software used for
the present Inner Detector and applying it to ITk geometries and contributed to
the studies to make a physics case for a large eta extension (up to $\eta = 4$)
of the ITk. The aforementioned software package is still in development and I
represent the ITk responsible.

I further studied the effects of a new algorithm called TIDE (Tracking in dense
environments), which was developed for Run-2. In using this new algorithm for
ITk geometries, I help making decisions on the not yet fully defined detector
layout. I supervise a PhD student in studying efficiencies as a function of the
number of pixel and strip layers. All these studies are an important input to
the ITk community in defining the best layout of the inner tracker.

\end{statement}

\end{document}
