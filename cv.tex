%%% Set internalNotes to true to also print the internal notes; default is false
\newif\ifinternalNotes
%\internalNotestrue

\newif\ifreferences
%\referencestrue

\newif\ifresume
\resumetrue

\newif\ifstatement
\statementtrue

%%% Spacing used between publication entries
\def\spacingPubs{-12pt}
%%% Width of parbox used to right align dates
\def\parboxWidthOne{19pt}
\def\parboxWidthTwo{16pt}

\documentclass[]{cv} % Add 'print' as an option into the square bracket to remove colors from this template for printing

%%% Set PDF metadata
\hypersetup{colorlinks, pdfpagemode=UseOutlines, pdftitle=cv, pdfauthor={B. Schneider}}

%\addbibresource{bibliography.bib} % Specify the bibliography file to include publications

\begin{document}

\header{Basil }{Schneider}{} % Your name and current job title/field

%----------------------------------------------------------------------------------------
%	SIDEBAR SECTION
%----------------------------------------------------------------------------------------

\begin{aside} % In the aside, each new line forces a line break
  \section{contact}
  Fermilab
  PO Box 500
  Batavia IL 60510-5011
  ~
  +41 (0)78 710 37 55
  +1-630-827-2826
  ~
  \href{mailto:basil.schneider@cern.ch}{basil.schneider@cern.ch}
  \section{languages}
  German (native)
  English (fluent)
  French (moderate)
  Croatian (beginner)
  Norwegian (beginner)
  \section{computing}
  Linux
  C++, Python, Rust
  Root, RooFit, RooStats
  bash, sed, awk
  SQL
  git, svn
  HTML, CSS
  \LaTeX
  \section{besides physics}
  Cycling
  Hiking
  Music
\end{aside}

%----------------------------------------------------------------------------------------
%	EDUCATION SECTION
%----------------------------------------------------------------------------------------

\section{education \& employment}

\begin{entrylist}
%------------------------------------------------
  \entry
  {\parbox[t]{\parboxWidthOne}{Nov}\parbox[t]{\parboxWidthTwo}{\hfill '15} - now}
  {\textbf{Research Associate} at the \textbf{CMS experiment}}
  {Fermilab}
  {}
%------------------------------------------------
  \entry
  {\parbox[t]{\parboxWidthOne}{Nov}\parbox[t]{\parboxWidthTwo}{\hfill '14} - Oct '15}
  {\textbf{Postdoctoral Fellow} at the \textbf{ATLAS experiment}}
  {TRIUMF}
  {}
%------------------------------------------------
  \entry
  {\parbox[t]{\parboxWidthOne}{Jan}\parbox[t]{\parboxWidthTwo}{\hfill '11} - Jul '14}
  {\textbf{Ph.D.} at the \textbf{ATLAS experiment}}
  {University of Bern}
  {Ph.D. Thesis: A general approach to search for supersymmetry at the LHC by combining signal enhanced kinematic regions using the ATLAS detector (Supervisor:
  Prof. A. Ereditato)}
%------------------------------------------------
  \entry
  {\parbox[t]{\parboxWidthOne}{Sep}\parbox[t]{\parboxWidthTwo}{\hfill '08} - Mar '10}
  {\textbf{Master} of Science in \textbf{Theoretical Physics}}
  {ETH Zurich}
  {Master Thesis: The partition function of meromorphic conformal field theories at higher genus (Supervisor: Prof. M. Gaberdiel)}
%Overall grade: 5.13 (6 is the highest, 1 is the lowest; passmark is 4)}
%------------------------------------------------
  \entry
  {\parbox[t]{\parboxWidthOne}{Oct}\parbox[t]{\parboxWidthTwo}{\hfill '04} - Sep '08}
  {\textbf{Bachelor} of Science in \textbf{Experimental Physics}}
  {ETH Zurich}
  {Bachelor Thesis: Untersuchung der Cluster-Struktur von Elastomerpartikeln durch Simulation des Aggregationsvorganges und Partikelgr{\"o}ssen mittels dynamic
light scattering (Supervisor: Dr. Cornelius Gauer)}
%Overall grade: 5.07}
\entry
{\parbox[t]{\parboxWidthOne}{Sep}\parbox[t]{\parboxWidthTwo}{\hfill '04}}
{\textbf{Comprehensive entrance exam}}
{ETH Zurich}
{Exam at the level of a Matura}
%------------------------------------------------
\end{entrylist}

\section{awards}

\begin{entrylist}

  \entry
  {\parbox[t]{\parboxWidthOne}{Mar} '15}
  {\textbf{Faculty award winner of the University of Bern}}
  {}
  {Award for the best PhD thesis in physics at the University of Bern in the
  year 2014}

\end{entrylist}

\section{leadership}

\begin{entrylist}

  \entry
  {\parbox[t]{\parboxWidthOne}{Jan}\parbox[t]{\parboxWidthTwo}{\hfill '18} - now}
  {\textbf{SUSY Leptonic subgroup co-convener}}
  {}
  {}

  \entry
  {\parbox[t]{\parboxWidthOne}{Oct}\parbox[t]{\parboxWidthTwo}{\hfill '16} - May '17}
  {\textbf{Coordinator of the Single Lepton dPhi Analysis group}}
  {}
  {}

\end{entrylist}

\newpage
\section{conferences}

\begin{entrylist}

  \entrytwo
  {\parbox[t]{\parboxWidthOne}{Jul}\parbox[t]{\parboxWidthTwo}{\hfill '18}}
  {\textbf{Conference on Supersymmetry and Unification of Fundamental Interactions}}
  {Barcelona, Spain}
  {Speaker: ``Searches for chargino, neutralino and slepton production with CMS''}

  \entrytwo
  {\parbox[t]{\parboxWidthOne}{Oct}\parbox[t]{\parboxWidthTwo}{\hfill '17}}
  {\textbf{IEEE Nuclear Science Symposium and Medical Imaging Conference}}
  {Atlanta, GA, USA}
  {Poster: ``A new DAQ solution: \textit{otsdaq}''}

  \entry
  {\parbox[t]{\parboxWidthOne}{Aug}\parbox[t]{\parboxWidthTwo}{\hfill '17}}
  {\textbf{Meeting of the Division of Particles and Fields of the American
  Physical Society}\\}
  {Fermilab, Batavia, IL, USA}
  {Speaker: ``Searches for electroweakly produced supersymmetry with CMS''}

  \entry
  {\parbox[t]{\parboxWidthOne}{May}\parbox[t]{\parboxWidthTwo}{\hfill '17}}
  {\textbf{Phenomenology 2017 Symposium}}
  {Pittsburgh, PA, USA}
  {Speaker: ``Searches for supersymmetry in single or opposite-charged dilepton final states with CMS''}

  \entry
  {\parbox[t]{\parboxWidthOne}{Jun}\parbox[t]{\parboxWidthTwo}{\hfill '16}}
  {\textbf{49th Annual Fermilab Users Meeting}}
  {Fermilab, Batavia, IL, USA}
  {Poster: ``Characterization of the pixel ASIC with a laser beam in the Outer Tracker upgrade of the CMS detector''}

  \entry
  {\parbox[t]{\parboxWidthOne}{May}\parbox[t]{\parboxWidthTwo}{\hfill '15}}
  {\textbf{Mitchell Workshop on Collider and Dark Matter Physics}\\}
  {Texas A\&M University, College Station, TX, USA}
  {Speaker: ``Supersymmetry searches in ATLAS''}

  \entry
  {\parbox[t]{\parboxWidthOne}{May}\parbox[t]{\parboxWidthTwo}{\hfill '13}}
  {\textbf{1\textsuperscript{st} LHC Physics Conference (LHCP)}}
  %{\href{https://cds.cern.ch/record/1555743}{ATL-PHYS-SLIDE-2013-350}}
  {Barcelona, Spain}
  {Poster: ``Search for direct production of charginos and neutralinos in events with three
    leptons and missing transverse momentum in 21 fb\textsuperscript{-1} of pp collisions at $\sqrt{\mathsf{s}}$ = 8 TeV with the ATLAS
  detector''}

  \entry
  {\parbox[t]{\parboxWidthOne}{Jun}\parbox[t]{\parboxWidthTwo}{\hfill '12}}
  {\textbf{Swiss Physical Society}}
  {ETH, Zurich, Switzerland}
  {Speaker: ``New Optical receiver modules for the insertable B-Layer at the ATLAS project''}

  \entry
  {\parbox[t]{\parboxWidthOne}{Jun}\parbox[t]{\parboxWidthTwo}{\hfill '11}}
  {\textbf{Physics at LHC}}
  %{\href{https://cds.cern.ch/record/1371922}{ATL-PHYS-SLIDE-2011-423}}
  {Perugia, Italy}
  {Poster: ``SUSY Searches at ATLAS in Multilepton Final States with Jets and Missing Transverse Energy''}

  \entry
  {\parbox[t]{\parboxWidthOne}{Jun}\parbox[t]{\parboxWidthTwo}{\hfill '11}}
  {\textbf{Swiss Physical Society}}
  {EPF, Lausanne, Switzerland}
  {Speaker: ``Insertable b-Layer: A new layer for the ATLAS detector at CERN''}

\end{entrylist}

\section{journal publications}
\begin{entrylist}

  \entry
  {}
  {I am co-author of 475 ATLAS publications and 111 CMS publications;\\
  for a full list, see \\
    \href{http://inspirehep.net/author/profile/B.Schneider.1}{http://inspirehep.net/author/profile/B.Schneider.1}\\
  Publications with substantial contributions from me:}
  {}
  {\vspace*{\spacingPubs}}

  \entry
  {2018}
  {Performance of the Prototype CBC3-based Outer Tracker Module for the Phase-2
  Upgrade of CMS}
  {in preparation}
  {\vspace*{\spacingPubs}}

  \entry
  {2018}
  {Performance of Prototype Silicon Detectors for the Outer Tracker for the
  Phase-2 Upgrade of CMS}
  {in preparation}
  {\vspace*{\spacingPubs}}

  \entry
  {\parbox[t]{\parboxWidthOne}{Jan}\parbox[t]{\parboxWidthTwo}{\hfill '18}}
  {Search for new physics in events with two soft oppositely charged leptons and missing transverse momentum in proton-proton collisions at $\sqrt{\mathsf{s}}$~=~13~TeV}
    {\href{https://www.sciencedirect.com/science/article/pii/S037026931830426X}{10.1016/j.physletb.2018.05.062}}
  {\vspace*{\spacingPubs}}

  \entry
  {\parbox[t]{\parboxWidthOne}{Sep}\parbox[t]{\parboxWidthTwo}{\hfill '17}}
  {Search for supersymmetry in events with one lepton and multiple jets exploiting the angular correlation between the lepton and the missing transverse momentum in proton-proton collisions at $\sqrt{\mathsf{s}}$~=~13~TeV\\}
    {\href{http://dx.doi.org/10.1016/j.physletb.2018.03.028}{10.1016/j.physletb.2018.03.028}}
  {\vspace*{\spacingPubs}}

  \entry
  {\parbox[t]{\parboxWidthOne}{Sep}\parbox[t]{\parboxWidthTwo}{\hfill '16}}
  {Search for supersymmetry in events with one lepton and multiple jets in proton-proton collisions at $\sqrt{\mathsf{s}}$~=~13~TeV}
    {\href{https://journals.aps.org/prd/abstract/10.1103/PhysRevD.95.012011}{Phys. Rev. D 95, 012011 (2017)}}
  {\vspace*{\spacingPubs}}

  %newpage
  \end{entrylist}
  \begin{entrylist}

  \entry
  {\parbox[t]{\parboxWidthOne}{Sep}\parbox[t]{\parboxWidthTwo}{\hfill '15}}
  {Search for the electroweak production of supersymmetric particles in
    $\sqrt{\mathsf{s}}$~=~8~TeV pp collisions with the ATLAS detector\\}
    {\href{https://journals.aps.org/prd/abstract/10.1103/PhysRevD.93.052002}{Phys. Rev. D 93, 052002 (2016)}}
  {\vspace*{\spacingPubs}}

  \entry
  {\parbox[t]{\parboxWidthOne}{May}\parbox[t]{\parboxWidthTwo}{\hfill '14}}
  {Search for supersymmetry in events with four or more leptons in $\sqrt{\mathsf{s}}$~=~8~TeV pp collisions with the ATLAS detector\\}
  {\href{https://journals.aps.org/prd/abstract/10.1103/PhysRevD.90.052001}{Phys. Rev. D. 90, 052001 (2014)}}
  {\vspace*{\spacingPubs}}

  \entry
  {\parbox[t]{\parboxWidthOne}{Feb}\parbox[t]{\parboxWidthTwo}{\hfill '14}}
  {Search for direct production of charginos and neutralinos in events with three leptons and missing transverse momentum in $\sqrt{\mathsf{s}}$~=~8~TeV pp collisions with the ATLAS detector}
{\href{https://link.springer.com/article/10.1007\%2FJHEP04\%282014\%29169}{JHEP04(2014)169}}
{\vspace*{\spacingPubs}}

    \entry
    {\parbox[t]{\parboxWidthOne}{Aug}\parbox[t]{\parboxWidthTwo}{\hfill '12}}
    {Search for direct production of charginos and neutralinos in events with three leptons and missing transverse momentum in $\sqrt{\mathsf{s}}$~=~7~TeV pp collisions with the ATLAS detector}
  {\href{https://www.sciencedirect.com/science/article/pii/S037026931201204X?via\%3Dihub}{Phys.Lett. B718 (2013) 841-859}}
  {\vspace*{\spacingPubs}}

\end{entrylist}

\section{proceedings}

\begin{entrylist}

  \entry
  {\parbox[t]{\parboxWidthOne}{Jun}\parbox[t]{\parboxWidthTwo}{\hfill '13}}
  {Search for direct production of charginos and neutralinos in events with
  three leptons and missing transverse momentum in 21 fb\textsuperscript{-1} of
  pp collisions at $\sqrt{\mathsf{s}}$ = 8 TeV with the ATLAS detector\\}
  {\href{https://doi.org/10.1051/epjconf/20136020040}{EPJ Web of Conferences 60, 20040 (2013)}}
  {\vspace*{\spacingPubs}}

  \entry
  {\parbox[t]{\parboxWidthOne}{Nov}\parbox[t]{\parboxWidthTwo}{\hfill '11}}
  {The ATLAS IBL BOC Demonstrator\\}
  {\href{http://ieeexplore.ieee.org/xpl/mostRecentIssue.jsp?punumber=6144196}{Proceedings, 2011 IEEE Nuclear Science Symposium and Medical Imaging Conference}}
  {\vspace*{\spacingPubs}}

  \entry
  {\parbox[t]{\parboxWidthOne}{Oct}\parbox[t]{\parboxWidthTwo}{\hfill '11}}
  {SUSY Searches at ATLAS in Multilepton Final States with Jets and Missing Transverse Energy}
  {\href{http://www.slac.stanford.edu/econf/C1106061}{Proceedings, Physics at LHC 2011}}
  {\vspace*{\spacingPubs}}

\end{entrylist}

\section{public notes}

\begin{entrylist}

  \entry
  {2018}
  {Search for Supersymmetry at the HL-LHC with the Upgraded CMS Detector}
  {in preparation}
  {\vspace*{\spacingPubs}}

  \entry
  {\parbox[t]{\parboxWidthOne}{Apr}\parbox[t]{\parboxWidthTwo}{\hfill '18}}
  {The Phase-2 Upgrade of the CMS Endcap Calorimeter -- Technical Design Report}
  {\href{https://cds.cern.ch/record/2293646}{CERN-LHCC-2017-023}}
  {\vspace*{\spacingPubs}}

  \entry
  {\parbox[t]{\parboxWidthOne}{Sep}\parbox[t]{\parboxWidthTwo}{\hfill '17}}
  {The Phase-2 Upgrade of the CMS Barrel Calorimeters -- Technical Design Report}
  {\href{https://cds.cern.ch/record/2283187}{CERN-LHCC-2017-011}}
  {\vspace*{\spacingPubs}}

  \entry
  {\parbox[t]{\parboxWidthOne}{Jun}\parbox[t]{\parboxWidthTwo}{\hfill '17}}
  {The Phase-2 Upgrade of the CMS Tracker -- Technical Design Report\\}
  {\href{https://cds.cern.ch/record/2272264}{CERN-LHCC-2017-009}}
  {\vspace*{\spacingPubs}}

  \entry
  {\parbox[t]{\parboxWidthOne}{Dec}\parbox[t]{\parboxWidthTwo}{\hfill '16}}
  {Search for new physics in the compressed mass spectra scenario using events with two soft opposite-sign leptons and missing transverse momentum at $\sqrt{\mathsf{s}}$~=~13~TeV}
  {\href{https://cds.cern.ch/record/2205866}{CMS PAS SUS-16-025}}
  {\vspace*{\spacingPubs}}

  \entry
  {\parbox[t]{\parboxWidthOne}{Aug}\parbox[t]{\parboxWidthTwo}{\hfill '16}}
  {Search for supersymmetry in events with one lepton and multiple jets in proton-proton collisions at $\sqrt{\mathsf{s}}$~=~13~TeV in 2016}
  {\href{https://cds.cern.ch/record/2204932}{CMS PAS SUS-16-019}}
  {\vspace*{\spacingPubs}}

  \entry
  {\parbox[t]{\parboxWidthOne}{Jul}\parbox[t]{\parboxWidthTwo}{\hfill '15}}
  {First look at proton proton collision data at $\sqrt{\mathsf{s}}$~=~13~TeV in
  preparation for a search for squarks and gluinos in events with missing
  transverse energy, jets, and an isolated electron or muon}
  {\href{https://cds.cern.ch/record/2037906}{ATL-PHYS-PUB-2015-029}}
  {\vspace*{\spacingPubs}}

  \entry
  {\parbox[t]{\parboxWidthOne}{Mar}\parbox[t]{\parboxWidthTwo}{\hfill '15}}
  {Expected sensitivity studies for gluino and squark searches using the early
  LHC 13 TeV Run-2 dataset with the ATLAS experiment\\}
  {\href{https://cds.cern.ch/record/2002608/}{ATL-PHYS-PUB-2015-005}}
  {\vspace*{\spacingPubs}}

  \entry
  {\parbox[t]{\parboxWidthOne}{Jun}\parbox[t]{\parboxWidthTwo}{\hfill '14}}
  {A general approach to search for supersymmetry at the LHC by combining signal
  enhanced kinematic regions using the ATLAS detector (PhD thesis)\\}
  {\href{https://cds.cern.ch/record/1709233/}{CERN-THESIS-2014-056}}
  {\vspace*{\spacingPubs}}

  \entry
  {\parbox[t]{\parboxWidthOne}{Mar}\parbox[t]{\parboxWidthTwo}{\hfill '13}}
  {Search for supersymmetry in events with four or more leptons in 21 fb\textsuperscript{-1} of pp collisions at
  $\sqrt{\mathsf{s}}$ = 8 TeV with the ATLAS detector\\}
  {\href{https://cds.cern.ch/record/1532429}{ATLAS-CONF-2013-036}}
  {\vspace*{\spacingPubs}}

  %newpage
  \end{entrylist}
  \begin{entrylist}

  \entry
  {\parbox[t]{\parboxWidthOne}{Mar}\parbox[t]{\parboxWidthTwo}{\hfill '13}}
  {Search for direct production of charginos and neutralinos in events with three leptons and missing transverse momentum in 21
    fb\textsuperscript{-1} of pp collisions at $\sqrt{\mathsf{s}}$ = 8 TeV with the ATLAS detector}
    {\href{https://cds.cern.ch/record/1532426}{ATLAS-CONF-2013-035}}
    {\vspace*{\spacingPubs}}

    \entry
    {\parbox[t]{\parboxWidthOne}{Nov}\parbox[t]{\parboxWidthTwo}{\hfill '12}}
    {Search for direct production of charginos and neutralinos in events with three leptons and missing transverse momentum in 13.0
      fb\textsuperscript{-1} of pp collisions at $\sqrt{\mathsf{s}}$ = 8 TeV with the ATLAS detector}
      {\href{https://cds.cern.ch/record/1493493}{ATLAS-CONF-2012-154}}
      {\vspace*{\spacingPubs}}

      \entry
      {\parbox[t]{\parboxWidthOne}{Nov}\parbox[t]{\parboxWidthTwo}{\hfill '12}}
      {Search for Supersymmetry in events with four or more leptons in 13 fb\textsuperscript{-1} pp collisions at $\sqrt{\mathsf{s}}$
    = 8 TeV with the ATLAS detector}
    {\href{https://cds.cern.ch/record/1493492}{ATLAS-CONF-2012-153}}
    {\vspace*{\spacingPubs}}

\end{entrylist}

\section{organization}

\begin{entrylist}

  \entry
  {\parbox[t]{\parboxWidthOne}{Aug}\parbox[t]{\parboxWidthTwo}{\hfill '12}}
  {\textbf{Co-organizer of workshop: ATLAS SUSY Statistical Interpretations workshop}}
  {}
  {Wrap up lessons learned in previous round of publications and spot possible
  improvements for next round}

  \entry
  {\parbox[t]{\parboxWidthOne}{Sep}\parbox[t]{\parboxWidthTwo}{\hfill '11}}
  {\textbf{Co-organizer of outreach event: Nacht der Forschung}}
  {University of Bern}
  {Performing experiments in public and discussing results}

\end{entrylist}

\section{outreach}

\begin{entrylist}

  \entry
  {\parbox[t]{\parboxWidthOne}{Jan}\parbox[t]{\parboxWidthTwo}{\hfill '16}}
  {\textbf{Fermilab Open House}}
  {}
  {Explaining the purpose and the mission of Fermilab to the public}

  \entry
  {\parbox[t]{\parboxWidthOne}{Nov}\parbox[t]{\parboxWidthTwo}{\hfill '13} - now}
  {\textbf{Official ATLAS underground guide}}
  {}
  {Showing the ATLAS detector to the public during LHC shutdowns}

  \entry
  {\parbox[t]{\parboxWidthOne}{Mar}\parbox[t]{\parboxWidthTwo}{\hfill '12} - Mar
'13}
  {\textbf{Masterclasses}}
  {University of Bern}
  {Helping high school students performing measurements on real LHC data}

  \entry
  {\parbox[t]{\parboxWidthOne}{Sep}\parbox[t]{\parboxWidthTwo}{\hfill '11}}
  {\textbf{Nacht der Forschung}}
  {University of Bern}
  {Presenting LHC physics on a poster and answering questions of the public in a
  research outreach event at the University of Bern}

\end{entrylist}

\section{committees}

\begin{entrylist}

  \entry
  {\parbox[t]{\parboxWidthOne}{Feb}\parbox[t]{\parboxWidthTwo}{\hfill '18}}
  {Analysis Review Committee: Search for supersymmetry in events with two tau
  leptons and missing transverse momentum in $\sqrt{s}$~=~13~TeV proton-proton
  collisions with the CMS detector}
  {\href{}{to be submitted to JHEP}}
  {}

  \entry
  {\parbox[t]{\parboxWidthOne}{Jul}\parbox[t]{\parboxWidthTwo}{\hfill '17}}
  {Analysis Review Committee: Search for pair production of tau sleptons in
  $\sqrt{s}$~=~13~TeV pp collisions in the all-hadronic final state}
  {\href{https://cds.cern.ch/record/2273395}{CMS-PAS-SUS-17-003}}
  {}

  \entry
  {\parbox[t]{\parboxWidthOne}{Jan}\parbox[t]{\parboxWidthTwo}{\hfill '17}}
  {Analysis Review Committee: Search for physics beyond the standard model in
  events with two leptons of same sign, missing transverse momentum, and jets in
  proton-proton collisions at $\sqrt{s}$~=~13~TeV\\}
  {\href{https://link.springer.com/article/10.1140\%2Fepjc\%2Fs10052-017-5079-z}{Eur. Phys. J. C 77 (2017) 578}}
  {}

  \entry
  {\parbox[t]{\parboxWidthOne}{Jul}\parbox[t]{\parboxWidthTwo}{\hfill '16}}
  {Analysis Review Committee: Search for SUSY in same-sign dilepton events at 13 TeV}
  {\href{https://cds.cern.ch/record/2204929}{CMS-PAS-SUS-16-020}}
  {}

\end{entrylist}

\newpage
\section{teaching}

\begin{entrylist}

  \entry
  {\parbox[t]{\parboxWidthOne}{Jan}\parbox[t]{\parboxWidthTwo}{\hfill '11} - May '14}
  {\textbf{Lab Course}}
  {University of Bern}
  {Supervising and assisting Physics undergraduate students working on fundamental experiments in mechanics and electronics}

  \entry
  {\parbox[t]{\parboxWidthOne}{Jan}\parbox[t]{\parboxWidthTwo}{\hfill '11} - May '14}
  {\textbf{Physics for Biologists}}
  {University of Bern}
  {Assisting 1\textsuperscript{st} year Physics course}

  \entry
  {\parbox[t]{\parboxWidthOne}{Jul}\parbox[t]{\parboxWidthTwo}{\hfill '11} - May '14}
  {\textbf{Private lessons for high-school graduates}}
  {Interlink Schulberatung GmbH}
  {Private lessons in Mathematics, Statistics and Physics}

  \entry
  {\parbox[t]{\parboxWidthOne}{Jun}\parbox[t]{\parboxWidthTwo}{\hfill '08}}
  {\textbf{Exam preparation}}
  {ETH Zurich}
  {Exam preparation for 1\textsuperscript{st} year Physics and Mathematics students}

  \entry
  {2007/2008}
  {\textbf{Teaching assistant}}
  {ETH Zurich}
  {Teaching assistant for environmental science students in Calculus}

\end{entrylist}

\section{supervision}

\begin{entrylist}

  \entry
  {\parbox[t]{\parboxWidthOne}{Jul}\parbox[t]{\parboxWidthTwo}{\hfill '16} - Oct '16}
  {\textbf{David Jin}}
  {}
  {Summer Student at Fermilab, University of Chicago}

  \entry
  {\parbox[t]{\parboxWidthOne}{May}\parbox[t]{\parboxWidthTwo}{\hfill '16} - Aug '16}
  {\textbf{Christian Leefmans}}
  {}
  {Summer Student at Fermilab, Cornell University}

  \entry
  {\parbox[t]{\parboxWidthOne}{Dec}\parbox[t]{\parboxWidthTwo}{\hfill '14} - Nov '15}
  {\textbf{Felix Cormier}}
  {}
  {MSc student at CERN, University of British Columbia}

  \entry
  {\parbox[t]{\parboxWidthOne}{Nov}\parbox[t]{\parboxWidthTwo}{\hfill '14} - Nov '15}
  {\textbf{Matthew Gignac}}
  {}
  {PhD student at CERN, University of British Columbia}

  \entry
  {\parbox[t]{\parboxWidthOne}{Dec}\parbox[t]{\parboxWidthTwo}{\hfill '12} - Mar' 14}
  {\textbf{Benjamin Gerber}}
  {}
  {MSc student, University of Bern}

\end{entrylist}

\ifinternalNotes
\section{internal notes}

\begin{entrylist}

  \entry
  {\parbox[t]{\parboxWidthOne}{Jul}\parbox[t]{\parboxWidthTwo}{\hfill '15}}
  {First look at proton proton collision data at $\sqrt{\mathsf{s}}$ = 13 TeV in
  preparation for a search for squarks and gluinos in events with missing
  transverse energy, jets, and an isolated electron or muon}
{\href{https://cds.cern.ch/record/2034389}{ATL-COM-PHYS-2015-718}}
{\vspace*{\spacingPubs}}

  \entry
  {\parbox[t]{\parboxWidthOne}{May}\parbox[t]{\parboxWidthTwo}{\hfill '15}}
  {ATLAS Large eta task force report}
{\href{https://cds.cern.ch/record/2016324}{ATL-COM-UPGRADE-2015-013}}
{\vspace*{\spacingPubs}}

  \entry
  {\parbox[t]{\parboxWidthOne}{Feb}\parbox[t]{\parboxWidthTwo}{\hfill '15}}
  {Expected sensitivity of search for squarks and gluinos in events with
    isolated leptons, jets and missing transverse momentum at
    $\sqrt{\mathsf{s}}$ = 13 TeV with the ATLAS detector}
{\href{https://cds.cern.ch/record/1994195}{ATL-COM-PHYS-2015-133}}
{\vspace*{\spacingPubs}}

  \entry
  {\parbox[t]{\parboxWidthOne}{Jan}\parbox[t]{\parboxWidthTwo}{\hfill '15}}
  {Re-interpretations and combinations of electroweak limits in
    20.3 fb\textsuperscript{-1} pp collisions at $\sqrt{\mathsf{s}}$
  = 8 TeV with the ATLAS detector}
{\href{https://cds.cern.ch/record/1981548}{ATL-COM-PHYS-2015-011}}
{\vspace*{\spacingPubs}}

  \entry
  {\parbox[t]{\parboxWidthOne}{Dec}\parbox[t]{\parboxWidthTwo}{\hfill '13}}
  {Search for supersymmetry in events with four or more leptons in $\sqrt{\mathsf{s}}$ = 8 TeV pp collisions with the ATLAS
detector}
{\href{https://cds.cern.ch/record/1635455}{ATL-COM-PHYS-2013-1621}}
{\vspace*{\spacingPubs}}

\entry
{\parbox[t]{\parboxWidthOne}{Oct}\parbox[t]{\parboxWidthTwo}{\hfill '13}}
{Search for supersymmetry in events with three leptons and missing transverse momentum in 21 fb\textsuperscript{-1} pp collisions
at $\sqrt{\mathsf{s}}$ = 8 TeV with the ATLAS detector}
{\href{https://cds.cern.ch/record/1610185}{ATL-PHYS-INT-2013-021}}
{\vspace*{\spacingPubs}}

\entry
{\parbox[t]{\parboxWidthOne}{Jul}\parbox[t]{\parboxWidthTwo}{\hfill '13}}
{Search for supersymmetry in events with three leptons and missing transverse momentum in 20.3 fb\textsuperscript{-1} pp
collisions at $\sqrt{\mathsf{s}}$ = 8 TeV with the ATLAS detector}
{\href{https://cds.cern.ch/record/1558985}{ATL-COM-PHYS-2013-888}}
{\vspace*{\spacingPubs}}

\entry
{\parbox[t]{\parboxWidthOne}{Dec}\parbox[t]{\parboxWidthTwo}{\hfill '12}}
{Search for Supersymmetry in events with four or more leptons in 20.7~fb\textsuperscript{-1} pp collisions at $\sqrt{\mathsf{s}}$
= 8 TeV with the ATLAS detector}
{\href{https://cds.cern.ch/record/1501709}{ATL-COM-PHYS-2012-1819}}
{\vspace*{\spacingPubs}}

\entry
{\parbox[t]{\parboxWidthOne}{Dec}\parbox[t]{\parboxWidthTwo}{\hfill '12}}
{Search for Supersymmetry in events with four or more leptons in 13 fb\textsuperscript{-1} pp collisions at $\sqrt{\mathsf{s}}$ =
8 TeV with the ATLAS detector}
{\href{https://cds.cern.ch/record/1498627}{ATL-PHYS-INT-2012-096}}
{\vspace*{\spacingPubs}}

\entry
{\parbox[t]{\parboxWidthOne}{Dec}\parbox[t]{\parboxWidthTwo}{\hfill '12}}
{Search for supersymmetry in events with three leptons and missing transverse momentum in 13 fb\textsuperscript{-1} pp collisions
at $\sqrt{\mathsf{s}}$ = 8 TeV with the ATLAS detector}
{\href{https://cds.cern.ch/record/1498390}{ATL-PHYS-INT-2012-095}}
{\vspace*{\spacingPubs}}

\entry
{\parbox[t]{\parboxWidthOne}{Sep}\parbox[t]{\parboxWidthTwo}{\hfill '12}}
{SUSY Searches in the Final States with Three Leptons and Missing Transverse Momentum at ATLAS}
{\href{https://cds.cern.ch/record/1482141}{ATL-PHYS-INT-2012-059}}
{\vspace*{\spacingPubs}}

\end{entrylist}
\fi

%\section{experience}
%
%\begin{entrylist}
%
%  \entry
%  {\parbox[t]{\parboxWidthOne}{Jan}\parbox[t]{\parboxWidthTwo}{\hfill '11} - now}
%  {\textbf{Analysis: Supersymmetry}
%  \vspace{3pt}
%  \begin{itemize}
%    \item Binning and optimizing the three light lepton final state signal region
%    \item Signal region optimizations in three lepton final states with taus
%    \item Assembling and scripting a package of various scripts to facilitate limit computations; also used by other analyses
%    \item Statistical interpretations of results: Discovery p-values, exclusion contours, upper limits on model cross-sections
%    \item Simultaneous fit of the WZ background in control and signal regions
%    \item Optimization of distributed computing resources for limit calculations with pseudo-experiments
%    \item Presented the analyses at 2 SUSY-Approvals on behalf of the multilepton group
%    \item Assisting younger students
%  \end{itemize}}
%  {}
%  {\vspace*{-22pt}}
%
%  \entry
%  {\parbox[t]{\parboxWidthOne}{Jan}\parbox[t]{\parboxWidthTwo}{\hfill '11} - Dec '12}
%  {\textbf{Hardware: IBL}
%  \vspace{3pt}
%  \begin{itemize}
%    \item Building up the IBL activities at the University of Bern
%    \item Test of the optical readouts
%    \item Feasibility study of the usage of commercial products
%    \item Selecting best optical readout candidate based on experimental data
%    \item Rudimentary tests with FPGA programming in VHDL
%    \item Replacing dead optical transmitters in USA15
%    \item Presented studies at 2 IBL general meetings on behalf of the ATLAS Bern group
%  \end{itemize}}
%  {}
%  {\vspace*{-22pt}}
%
%\end{entrylist}

\ifreferences
\section{references}

\begin{entrylist}

  \entry
  {}
  {\textbf{Prof. Antonio Ereditato}}
  {PhD supervisor}
  {Laboratory for High Energy Physics, University of Bern\\+41 31 631 8566, \href{mailto:antonio.ereditato@cern.ch}{antonio.ereditato@cern.ch}}

  \entry
  {}
  {\textbf{Prof. Michele Weber}}
  {PhD supervisor}
  {Laboratory for High Energy Physics, University of Bern\\+41 31 631 5146, \href{mailto:weber@lhep.unibe.ch}{weber@lhep.unibe.ch}}

  \entry
  {}
  {\textbf{Dr. Jamie Boyd}}
  {SUSY convenor, external referee PhD thesis}
  {CERN\\+41 76 473 08 77, \href{mailto:jamie.boyd@cern.ch}{jamie.boyd@cern.ch}}

  \entry
  {}
  {\textbf{Dr. Anadi Canepa}}
  {SUSY Electroweak convenor}
  {TRIUMF, Canada's National Laboratory for Particle and Nuclear Physics\\+1 604-222-7330, \href{mailto:canepa@triumf.ca}{canepa@triumf.ca}}

  \entry
  {}
  {\textbf{Dr. Christina Potter}}
  {SUSY Electroweak convenor}
  {Department of Physics and Astronomy, University of Sussex\\+44 1273 873523, \href{mailto:christina.potter@sussex.ac.uk}{christina.potter@sussex.ac.uk}}

  \entry
  {}
  {\textbf{Dr. Tobias Flick}}
  {IBL Off-Detector Coordinator}
  {Detector Laboratory, Bergische Universit{\"a}t Wuppertal\\+49 202 439-2811, \href{mailto:flick@physik.uni-wuppertal.de}{flick@physik.uni-wuppertal.de}}

\end{entrylist}
\fi

\ifresume
\begin{resume}

\section{Analysis activities in ATLAS (up to 2015)}

\begin{itemize}
  \item I was the main analyzer of the search for charginos and neutralinos in
        events with three leptons and missing transverse momentum based on 8 fb$^{-1}$
        of data collected by ATLAS at $\sqrt{s}$~=~8~TeV (JHEP 04 (2014 169)). A large
        improvement in sensitivity was obtained, compared to previous analyses, by
        splitting the signal region in multiple bins depending on several kinematic
        variables and the multiplicity of taus in the final state. In addition to
        leading the analysis effort, I guided the optimization of the signal region
        for the final states with at least one tau. I was chosen to present the
        analysis in front of the collaboration for its approval.
  \item I carried out the statistical interpretations for four publications in
        ATLAS, all searching for supersymmetry in either three lepton or four lepton
        final states (Phys. Rev. D 93, 052002 (2016), Phys. Rev. D. 90, 052001 (2014),
        JHEP04 (2014) 169, Phys.Lett. B718 (2013) 841-859) and became the reference
        person for statistical interpretations in ATLAS SUSY EWK searches. In
        one these searches I also measured the most important background, WZ.
  \item I combined the three lepton electroweak SUSY search with a search in a
        final state with two leptons (JHEP 05 (2014) 071).
  \item During the Long Shutdown 1, I supervised a student and we released a
        public note (ATL-PHYS-PUB-2015-005) assessing the discovery potential of
        gluinos as a function of their mass and the integrated luminosity in the
        final state with one lepton.
\end{itemize}

\vspace{12pt}
\section{Analysis activities in CMS (since 2015)}

\begin{itemize}
  \item I thoroughly studied and compared the software packages that ATLAS and
        CMS use to carry out their statistical interpretations of analyses to
        understand the differences in features and scope.
  \item I carried out a phenomenological study to understand what parts of the
        SUSY phase space can explain the relic abundance of dark matter, but
        is not currently covered by SUSY searches.
  \item I led the search for strongly produced SUSY particles in final states
        with a single lepton to a publication (10.1016/j.physletb.2018.03.028).
        I was responsible for organizing and coordinating the analysis,
        optimizing the analysis strategy and presenting the analysis to the
        collaboration for approval. The optimization of the search strategy led
        to an improvement in the exclusion limit for the gluino mass of 100~GeV
        in the limit of low LSP masses. In addition I was responsible for the
        measurement of the multijet background.
  \item I am one of the main analyzers in the soft multilepton group that is
        targeting light natural higgsinos. I carried out phenomenological studies and
        added an interpretation, that could solve the hierarchy problem and provide an
        excellent candidate for dark matter. This analysis is expected to be one of
        the first SUSY searches to use the combined dataset of 2016 and 2017. I am the
        contact person for the publication that is submitted to PLB (arXiv:1801.01846
        [hep-ex]).
  \item I assessed the sensitivity of a SUSY search in a final state of two low
        momentum leptons for the HL-LHC with an integrated luminosity of 3000
        fb$^{-1}$ at a center-of-mass energy of 14~TeV and 200 additional pileup
        events. To develop the analysis itself, I contributed significantly to
        the validation of Delphes (JHEP 02 (2014) 057) for HL-LHC searches. This
        in turn enabled the development of searches for SUSY in the final states
        with two same-charge leptons and taus. These analyses will be included
        in the upcoming Yellow Report for the European Strategy for Particle
        Physics and documented in a PAS that is in preparation.
  \item I am the SUSY Leptonic co-convener, organizing biweekly meetings,
        coordinating and reviewing the effort of several analyses that either
        search for electroweakly produced SUSY or use at least two leptons in
        their final state.
\end{itemize}

\vspace{12pt}
\section{Detector R\&D for the Phase-1 upgrade and MC simulations for HL-LHC on
ATLAS (up to 2015)}

\begin{itemize}
  \item I set up the laboratory at the University of Bern that was used for the
        test of the optical receivers for the read out of the insertable b-layer
        (IBL), which is the additional pixel detector layer, closest to the
        interaction point, that was added in the ATLAS experiment during the
        Long Shutdown 1. I defined the acceptance criteria for the optical
        receivers, based on the reliability, sensitivity to the input light
        intensity and frequency range, and then used them to qualify the
        receivers finally installed in the cavern. I presented the results on
        this qualification in the ATLAS IBL general meetings. My work has helped
        establishing the group of the University of Bern as a leader for this
        kind of measurements in ATLAS.
  \item While at TRIUMF, I have contributed to the design of the Phase-2 tracker
        upgrade of the ATLAS detector with Monte Carlo studies of different
        geometries of the silicon inner detector (ITk), focusing in particular on
        the benefits, from the point of view of physics analyses, of a possible
        extension of the rapidity coverage of the tracker. I also studied the
        impact of using a new tracking algorithm originally developed for Run-2
        of ATLAS, called TIDE (Tracking in dense environments) on the proposed
        ITk layouts. Finally, I supervised a Ph.D. student who was studying the
        tracking efficiency as a function of the total number of pixel and strip
        layers in the ATLAS tracker. All these studies have been an important
        input to the ITk community in defining the best layout of the inner
        tracker.
\end{itemize}

\vspace{12pt}
\section{Detector R\&D for HL-LHC on CMS (since 2015)}

\begin{itemize}
  \item I have worked in two areas related to the R\&D program for the CMS
        Phase-2 Outer Tracker: the development of a data acquisition tool
        (\textit{otsdaq}) that could be used in all phases of the detector
        development and construction (from bench tests of individual components
        to data taking in a beam with multiple detectors), and hardware tests of
        prototypes of modules and of parts of modules.
  \item I measured the properties of the first macro-pixel prototype, the
        MaPSA-Light, that combined a silicon sensor with a dedicated readout
        chip. In CMS, these macro-pixel detectors will be attached to a strip
        detector to form the PS modules. Using a laser system to generate
        charges in the silicon sensor I performed efficiency and time-walk
        measurements for the first time in a MaPSA-Light prototype.
  \item Later I measured the properties of all the available (three) Phase-2 2S
        modules (these are modules with two silicon strip sensors and
        electronics that form spatial coincidences between hits in the two
        layers). The characterization studies of these modules led to the
        discovery of several flaws in the firmware, software and module design,
        all of which have been later fixed. The results from these tests have
        been included in the CMS Technical Design Report for the Phase-2
        tracking system.
  \item I have qualified all full-sized hybrids for the 2S modules (these are
        flexible circuits that are used to connect the two silicon strip layers
        in a 2S module) that have been commercially manufactured, prior to their
        use for the construction of prototype modules in different CMS member
        institutions.
  \item I have set up several read out systems at CERN and at Fermilab, both to
        read out Outer Tracker modules and hybrids. The read out system set up
        at Fermilab has later been used in the test beam, where we measured for
        the first time the response of version 3 of the CBC ASIC to a particle
        beam.
  \item I developed the Outer Tracker part of \textit{otsdaq} (off-the-shelf
        DAQ), a generic data acquisition tool for particle physics experiments.
        \textit{otsdaq} has been used in several test beams, both at CERN and at
        Fermilab. It provides a graphical user interface for simple and
        efficient data taking, live data quality monitoring and a graphical user
        interface to configure the detector.
  \item I participated in several test beams, by setting up the system and allow
        for efficient data taking. The data from these test beams will be used
        for a publication that is in preparation.
\end{itemize}

\end{resume}
\fi

\ifstatement
\begin{statement}

\section{research proposal}

The Large Hadron Collider (LHC) at CERN is the largest particle accelerator
built to date and one of the most daring projects accomplished by humankind. The
two largest experiments that are analyzing the data of the LHC have a
multi-purpose goal: ATLAS and CMS. Both these experiments found a particle in
2012 that was, and to date still is, compatible with the boson associated to the
Brout-Engler-Higgs mechanism. Besides the Higgs boson, many measurements have
been carried out with previously unseen precision and many searches for
particles beyond the Standard Model (SM) of particle physics have been carried
out.

The achievements of the LHC and its experiments are impressive. But so far, only
about 3 \% of the projected data has been taken. Starting with Run-1 in 2010,
data has been taken at a center-of-mass energy of 7 and 8 TeV. After a long
shutdown of 2 years, Run-2 started in 2015 at a center-of-mass energy of 13 TeV.
 Until the end of 2017, a total of about 80 fb$^{-1}$ of data has been taken. At
the end of Run-3, in 2023, a total of 300 fb$^{-1}$ is expected to be recorded
by CMS.  The LHC and its experiments will then undergo an upgrade to deliver and
record larger instantaneous luminosities. This is referred to as the Phase-2
upgrades, and the LHC will become the High-Luminosity LHC (HL-LHC). At the end
of the lifetime of the HL-LHC, a total of 3000 fb$^{-1}$ of data is expected.

\vspace{15pt}
\Large{}
\textbf{Physics}
\normalsize{}

Searches for new physics have so far not revealed any hint of physics beyond the
SM. However, it is clear that such new physics must exist. The SM is one of the
most successful theories humankind has ever developed, but the theory is not
complete. The major deficit of the SM is, that it does not include gravity. It
does also not provide a candidate for dark matter, a hypothetical particle that
could explain many measurements that we observe in the Universe, among others
the rotational curves of galaxies. Another shortcoming of the SM is the
naturalness problem: The higgs mass receives radiative corrections to its mass
that are many orders of magnitudes larger than the mass itself. This extremely
unlikely scenario is called the fine-tuning problem of the Higgs mass.

Before the advent of the LHC era, it was widely expected that Supersymmetric
particles would be found, even with little amount of data, as the lightest
strongly interacting particles were believed to be not much heavier than 1 TeV.
Supersymmetry (SUSY) is an intriguing theory. By effectively doubling the
particle content of the SM and exploiting the only Poincar\'{e} symmetry left
that has not yet revealed itself in nature, many shortcomings of the SM can be
explained.

The three popular shortcomings of the SM that can be solved with SUSY are: The
naturalness problem, the unification of all interaction strengths at the grand
unification energy (GUT scale) and the existence of a dark matter candidate.
However, the first two of these three so called shortcomings, the naturalness
problem and the unification of energies at the GUT scale, are maybe not real
shortcomings of the theory.  They only arise because we have a certain pretense
for our theories to be beautiful. Fine-tuning might exist and we might just live
in a world where the Higgs mass is very small, compared to its corrections, by
accident. And there is no reason why we should expect the interaction strengths
to unify at a certain energy.

The unsuccessful unification of quantum field theory with general relativity and
the non-existence of a dark matter candidate in the SM, on the other hand,
cannot be explained away. There must either be a correction to the theory of
general relativity, or an additional particle that we haven't accounted for yet.
As experimental physicists, it is one of our largest priorities, to search for a
dark matter candidate, be it in the context of SUSY or in pure dark matter
searches.

The LHC program has also lead to the insight that despite many carefully planned
searches, no SUSY particles have been found to date. There are basically three
ways to explain this null-result: Either SUSY is a symmetry that is not observed
in nature, or SUSY particles are heavy and out of reach by the LHC, or these
particles are elusively hiding behind large backgrounds and systematic
uncertainties, or in final states that are difficult to trigger on and
reconstruct. We cannot know which of these scenarios answers the question why we
have not found any SUSY particles. However, while we cannot fully eliminate
either of the answers (except by indeed observing SUSY), we can at least refine
our searches to make sure to not miss any SUSY particles in our data.

The theoretical community shows many ways out of the perceived tension between
SUSY theory and its non-observation. However, we also have to consider that one
of the favorite theories of the particle physics community might not be realized
in nature. It is important, though, that CMS continues a broad program for
searches beyond the SM, be it in the context of SUSY or not. The SUSY search
community in CMS has a wide spectrum of knowledge for searches. It is important
to keep this know-how and extend searches to less obvious cases. More emphasis
should be laid on less common search strategies. And we should not forget, that
so far we only analyzed around 3 \% of the data we expect to receive from the LHC.

The current trigger strategies need to be revisited. Currently, most SUSY
analyses rely on a trigger that is firing when a sufficient large amount of
missing transverse momentum is observed in an event. This has not yet led to the
observation of any new particles. Instead we have to introduce triggers that
include other objects, like for example the two low momentum muons that my SUSY
search in a final state with two low momentum leptons commissioned. Other
possibilities should be explored more often already at trigger level, such as
angular information or the possibility of using machine learning algorithms. A
good trigger strategy is crucial: If SUSY is hiding and we do not trigger its
events, we will not find it. Including trigger information from the Outer
Tracker during Phase-2 of data taking, will open up many possibilities for new
trigger strategies, despite larger rates due to the larger instantaneous
luminosity. As an example, it is currently very difficult to trigger on very low
momentum electrons, since they are completely dominated by instrumental noise in
the electronic calorimeters.

Currently CMS uses data parking and scouting. Parked data are data recorded by
firing a trigger that is either a loose version of an existing trigger or a
completely different trigger that does not overlap with any others. This data is
parked, meaning they are not reconstructed immediately but only when spare
computing time is available, for example during a long shutdown. In data
scouting, trigger thresholds are lowered as well. To control the bandwidth used
by these triggers, only a fraction of information is stored on tape. Currently,
non of the SUSY searches use either parked or scouted data. Exploiting these
strategies can lead to a substantial increase of sensitivity, especially in the
regions of parameter space that are relatively difficult to access.

Also on the reconstruction side we have to refine our analyses. There already
has been a lot of work done that aims in this direction. B-jet finding
algorithms have been improved lately, using a deep neural network to enhance the
efficiency for finding b-jets at the same mistagging rate. Similar approaches
are now explored for other objects, like top-tagging or W-tagging. These efforts
should be strengthened and exploited in all analyses where its use makes sense.

High energy physics is no longer the field where the most data is analyzed.
Other fields in computer science now analyze data sets that are larger than
ours. This has led to many advancements in the field of big data processing.
Machine learning algorithms have been improved and simplified. Languages and
packages exist, that can be relatively easily applied to our field. We have to
exploit these technical advancements and further improve the use of the latest
technologies. Keeping pace with these developments and an increased exchange
with experts from other fields can boost our sensitivity as well. There are
efforts ongoing in machine learning, but other parts of technology have a
similar potential. The programming language Rust can serve as an example here,
since it inherently features concurrency and memory safety, without compromising
speed. Moving away from C++ to Rust could lead to more efficient software that
is easier to maintain.

We also have to make sure that we systematically cover all possible final
states. For example, we currently have a search for electroweakly produced SUSY
particles that decay via a W and a Z gauge boson. The final states considered
are in two oppositely charged leptons, three leptons and two oppositely charged
low momentum leptons. I am part of the analysis groups that exploits the final
state with two low momentum leptons and we are in the process of expanding the
final state to three low momentum leptons. There is large room for improvement
here: Analyses should make sure to cover the whole parameter space. This is not
the case right now, as while there are searches for two high momentum leptons as
well as two low momentum leptons, there is no search that covers a high momentum
and a low momentum lepton. This could be achieved relatively easily and I have
defined it as one of my goals during my co-convenership of the SUSY Leptonic
subgroup, to make sure that these corner cases are well covered. Second, a final
state with leptons had better sensitivity in this model up till now, but this is
about to change. Once exclusion limits for electroweak SUSY particles are pushed
to high values, purely hadronic final states might become more sensitive.
Leptonic final states have the advantage of less background, hadronic ones have
the higher branching ratios. Once the SUSY particles become heavy enough,
background can be suppressed in the hadronic final states by requiring high
momentum jets. At a certain point, the hadronic search will take over in
sensitivity from the leptonic searches. We have to explore this new decay mode.

While SUSY theory provides a dark matter candidate, we should also strengthen
the pure dark matter searches. There exist already many dark matter searches and
some of the pure dark matter searches are covered by SUSY searches. But SUSY
often predicts a final state with several high momentum objects, while there is
usually less activity in a pure dark matter search. We have to understand what
phase space of pure dark matter searches is not covered yet. Dark matter can be
searched for in direct detection, indirect detection, or collider experiments.
We need to communicate with our colleagues from other fields, and of course also
with the theory community, to better understand how we can extend our dark
matter searches to cover the maximal possible parameter space.

\vspace{15pt}
\Large{}
\textbf{Phase-2 Inner Tracker}
\normalsize{}

The HL-LHC upgrade implies extreme challenges for the design and the operation
of the tracker in terms of radiation tolerance of sensors and readout
electronics. This is especially true for the inner part of the tracker, equipped
with pixel sensors: the Inner Tracker. Since I have been testing the modules of
the Outer Tracker extensively both on a benchtop system and in several test
beams, I can bring my expertise on setting up test stands and testing detector
components from the CMS Phase-2 Outer Tracker to the pixel system. For the
module qualification the plan of CMS is to use dedicated systems. Based on the
experience with the previous CMS pixel detectors and with the Outer Tracker,
there is a lot to gain from establishing early test stands where the entire
chain, from the module to the electrical to optical conversion to the DAQ
backend is tested in conditions as close as possible to those expected for the
final detector.

It is important to have a working chain to read out existing module components
as soon as possible, then extending it with the components as they become
available.  This should happen in close collaboration with ETH Zurich and PSI.
Even though these institutes have different interests and responsibilities, all
groups will benefit from an efficient read-out chain. The software needed to
communicate with the components needs to be developed in parallel and should be
done in close collaboration with the Outer Tracker DAQ group, since a lot of
expertise is in this group and already many lessons have been learned in the
past years.

I want to pioneer the use of \textit{otsdaq} for the pixel detector read-out
system, this can have an impact beyond the extended pixel detector.
\textit{otsdaq} is a generic and highly flexible and scalable DAQ solution for
high energy physics experiments. I already adopted \textit{otsdaq} to Outer
Tracker modules and integrated it in the Outer Tracker DAQ. I intend to do the
same for the Inner Tracker hardware.

After getting a broad overview over the current status and the involvement of
the group at the University of Zurich in the Inner Tracker upgrade, I want to
take over a leading role in the on-detector electronics and the optical data
links. For the on-detector electronics I can bring in my expertise from the
Outer Tracker upgrade, and on optical data links I worked in the IBL upgrade in
ATLAS. As time passes I also want to become a leader in the construction of the
mechanics and the integration of the entire detector, and later in commissioning
and installation.

In the long term I also would like to understand if an upgrade of the current
pixel system design, that goes further than what is currently described in the
Technical Design Report is feasible.

\newpage
\section{research experience}

I have been a member of the ATLAS Collaboration between 2011 and 2015. Since
2015 I am a member of the CMS Collaboration.

My research program in both experiments has been driven by the search for SUSY
particles. I search for natural SUSY that could solve the hierarchy problem, as
well as providing a dark matter candidate that could explain the relic abundance
of dark matter particles that we observe in the Universe. I have a strong
background in phenomenological SUSY. With this knowledge I can devise searches
that are not covered by standard searches. I am also interested in understanding
what sensitivity we can expect to have in searches beyond the SM with the
Phase-2 detector at the HL-LHC, and later eventually at the HE-LHC.

An essential part for a general purpose detector at the LHC, such as ATLAS or
CMS is a robust tracking system to allow best possible position resolution, an
efficient matching of tracks to vertices and a performant track separation. I
did detector R\&D for the insertable b-layer (IBL) upgrade and optimized the
detector design, using robustness of physics observables for the Phase-2 upgrade
on ATLAS. In CMS I tested several components of module prototypes and developed
a data acquisition tool (\textit{otsdaq}) that could be used in all phases of the
detector development and operation.

\vspace{15pt}
\Large{}
\textit{SUSY Leptonic Co-Convener in CMS since January 2018}
\normalsize{}

Since January 2018, I am the co-convener of the SUSY Leptonic subgroup in CMS.
The Leptonic subgroup is specialized in SUSY searches with at least two leptons,
or targeting electroweak production processes. As a convener, I facilitate and
review several different analyses. I co-organize the biweekly meetings, in which
new analysis developments are discussed, analyses are reviewed and approved. I
read the documentation that the analysts provide, review analysis strategies and
make suggestions about how they can be improved. I also read the early paper
drafts that are put forward by the analysts. Once we agree that a certain
analysis is in a robust state, we allow the previously blind analyses to unblind
their signal regions. When an analysis gets approved, the review will be passed
on the the Analysis Review Committee. From this point on, we follow the analyses
and assist them in every way possible to efficiently lead them to a publication.
As Co-convener I will also co-lead one of the sessions at the yearly SUSY
workshop.

\vspace{15pt}
\Large{}
\textit{SUSY Search in two soft oppositely charged lepton final states in CMS}
\normalsize{}

A higgsino-like electroweakino is the most important ingredient to control the
mass of the Higgs boson. Its mass is the only SUSY parameter that enters the
Higgs boson mass already at tree level. A pure higgsino state leads to four
particles, two neutral and two charged ones, that are almost mass degenerate. If
they are not accompanied by other light SUSY particles, their detection will
pose a challenge, since the small mass splitting leads to low momentum objects.
These objects are difficult to reconstruct and trigger on. To be able to control
the large background that appears in searches with low momentum objects, our
search strategy targets leptonic decays of the neutral higgsinos. In the search
for two low momentum oppositely charged leptons, we require triggers to either
fire because of their large missing transverse momentum, or because of two very
low momentum muons, that additionally have a small amount of missing transverse
momentum. The low momentum muons trigger has been specifically introduced by our
analysis group. This search strategy was specifically chosen to target
higgsinos, but it is also sensitive to other compressed states, like
chargino-mediated stop decays.

In our SUSY searches, we usually use Simplified Models that include only the
minimal particle content necessary, i.e. the relevant particle masses are the
only free parameters. To a certain degree this lets the experimenter decouple
the analysis from the theory involved. However, it is not a priori clear that
Simplified Models cover all corners of a realistic extension of the SM. To
understand if the Simplified Models that we used in our search indeed cover the
higgsinos in a more generic SUSY model, I developed a more generic model for
additional interpretation. In the current paper, that is using the full 2016
dataset and is now submitted to PLB (arXiv:1801.01846 [hep-ex]) I am the contact
person and as such reviewed the paper draft in view of the comments we received
from the CMS Collaboration, and also from the journal referees.  

We intend to further extend our search regions, by including a three low
momentum lepton final state and enhance our sensitivity for signals with a
displaced vertex. We plan to have a PAS ready by ICHEP 2018, including the full
2016 and 2017 datasets. We expect to be one of the first analyses to publish a
SUSY search with both the 2016 and 2017 datasets combined. A journal publication
is planned to follow shortly thereafter.

\vspace{15pt}
\Large{}
\textit{Upgrade study to assess the sensitivity to SUSY models at the HL-LHC}
\normalsize{}

The HL-LHC upgrade implies extreme challenges for the design of the detector,
but also for the reconstruction and identification of particles. With the
increase of the instantaneous luminosity, more and more soft proton-proton
collisions (pileup) will be contaminating the hard process we are interested in.
The expected average number of additional pileup events will reach 200 on
average during the HL-LHC. This unprecedented large number of simultaneous
collisions will pose challenges, especially for soft objects, as for example in
the search for SUSY in two soft oppositely charged lepton final states. It is
therefore extremely important to study how much the improved detector and the
large dataset benefits the search and how much the additional pileup events hurt
the sensitivity.

I carried out a study to assess the sensitivity of a search for higgsinos with
two soft leptons and missing transverse momentum with the Phase-2 detector in an
environment with up to 200 pileup events. Since we need a large number of
background Monte Carlo events for a search with 3000 fb$^{-1}$ and since SUSY
searches additionally require a large number of signal events, both the CMS full
and fast detector simulation take too much computing time to be produced for a
sensitivity study. We therefore rely on Delphes, a fast multipurpose detector
response simulation (JHEP 02 (2014) 057).

I have compared the performance of the CMS Full Detector Simulation with the
ones from Delphes, measured reconstruction and identification efficiencies in
FullSim samples and improved the Delphes detector simulation to match the one
obtained by FullSim. I assessed the performance of the missing transverse
momentum in both detector simulations and compared how much they degrade
compared to Run-2, due to additional pileup events.

My validation studies on Delphes objects in turn enabled the search for
wino-like electroweakinos in a final state with two leptons of same charge and
searches for staus, both in hadronic and leptonic decay channels. The two
searches for electroweakinos will be combined in a larger framework: the
radiatively driven natural SUSY (10.1007/JHEP12(2013)013).

The results of the studies for HL-LHC that I have performed have been included
in the TDRs for the upgrade of the Barrel calorimeters (CERN-LHCC-2017-011) and
for the HGCAL (CERN-LHCC-2017-023). These studies, and the other ones that have
benefited from my improvements to Delphes, will also be included in a CERN
Yellow Report on the physics reach of the HL-LHC, that will also be used as
input document for the European Strategy Planning for Particle Physics. Prior to
their inclusion in the Yellow Report these analyses will also be made public as
CMS PAS.

Given my leadership role for SUSY analyses within the CMS HL-LHC studies group,
I have also been invited to give presentations at the Future Higgs Workshop at
CERN and the HL/HE-LHC Meeting at FNAL (the latter I had to decline due to
conflicting commitments).

\vspace{15pt}
\Large{}
\textit{SUSY Search exploiting angular information in single lepton final states in CMS}
\normalsize{}

The transition from 8 to 13 TeV in the center-of-mass energy of the LHC offered
an unprecedented increase in sensitivity for colored SUSY particles in the
vicinity of 1.5 TeV.

During the Long Shutdown 1, when I still was a member of the ATLAS
collaboration, I supervised a student and we released a public note
(ATL-PHYS-PUB-2015-005) assessing the discovery potential of gluinos as a
function of their mass and the integrated luminosity in the final state with one
lepton. We found that with already a small amount of data, we could surpass the
exclusion limits from the searches carried out at a center-of-mass energy at 8
TeV and given the right mass of the gluinos, we could have expected a discovery
with a relatively small dataset.

After my transition to CMS and Fermilab, I continued to work on this analysis in
the same final state. At the beginning of Run-2, I focused on light gluinos,
considering both decays mediated by light stops or light squarks. In this search
in a final state with exactly one lepton, I exploited the angle between the
reconstructed W and the lepton to search for SUSY events. I coordinated the
analysis by organizing the group meetings, assessing the analysis strategy and
leading the group to a publication in PLB (10.1016/j.physletb.2018.03.028).

I measured the QCD background and assessed the systematic uncertainty on its
yields. I optimized the binning of the signal of this search, to maximize the
sensitivity for a given signal. Due to my optimizations, the sensitivity
improved by ~100 GeV for heavy gluinos in the limit of massless neutralinos. I
was chosen to present the analysis in front of the collaboration for approval.

\vspace{15pt}
\Large{}
\textit{SUSY Search in three lepton final states in ATLAS}
\normalsize{}

During my PhD I searched for electroweakly produced SUSY in final states with
three leptons. In the search we considered gauge boson as well as light sleptons
mediated decays of associated neutralino-chargino production.

A large improvement in sensitivity was achieved by splitting the signal region
in multiple bins depending on several kinematic variables, such as the missing
transverse momentum, the invariant mass of a lepton pair and the transverse
mass. This was one of the first SUSY searches in ATLAS that introduced such a
binning of the signal region covering a large parameter space that is sensitive
to many different SUSY signals. By exploiting the difference in kinematics in
the individual signal regions I improved the sensitivity in the gauge boson
mediated decay by 180 GeV, and in the slepton mediated decay by 200 GeV.

In this search we also included final states with taus and interpreted results
in associated neutralino-chargino production with stau mediated decays. I also
optimized the signal regions in final states of three leptons including one tau.

I was also responsible for the statistical interpretations of results by
calculating discovery p-values, exclusion contours and upper limits on model
cross-sections. I helped implementing a data driven estimate of the most
important background, WZ, in the three lepton final state. I also carried out
the statistical interpretations of a search for R-Parity violating models in a
final state with four leptons. This effort has led to a total of 4 papers (Phys.
Rev. D 93, 052002 (2016), Phys. Rev. D. 90, 052001 (2014), JHEP04(2014)169,
Phys.Lett. B718 (2013) 841-859), of which I have been a driving analyst. During
this process I became the reference person for statistical interpretations in
ATLAS SUSY EWK searches

\vspace{15pt}
\Large{}
\textit{Electrical tests of several prototypes of the CMS Phase-2 Outer Tracker upgrade}
\normalsize{}

To cope with the large radiation and the high number of pileup events expected
at the HL-LHC, the CMS detector will need to be upgraded. One of the most
important aspects of the CMS HL-LHC upgrade is the complete replacement of the
tracking system. There are multiple reasons for this. First of all the
performance of the current detector will start degrading due to radiation
damage, and the current detector could not operate efficiently for a time long
enough. The new silicon tracker needs to be able to withstand much higher doses
compared to the current one. Then, an increase in granularity and the capability
of resolving tracks that are close in space are required to cope with the
increase in the number of pileup collisions, which is expected to reach up to
200 per bunch crossing. To extend the physics reach of CMS an increase in the
acceptance of the tracking volume up to |eta|=4 is also necessary. Finally, the
trigger thresholds for single leptons should not increase relative to the
current values in order to retain, for example, sensitivity to SUSY final states
with low momentum objects. This requires the use of tracking information already
at the first level of the trigger system.

In order to provide the tracking information for the L1 trigger decision the CMS
Phase-2 outer tracking system is designed in such a way that local track
segments are built in doublets of closely spaced sensors that form modules. Two
kind of modules are foreseen for the CMS Outer Tracker: a module with two strip
sensors (2S) and a module with a strip sensor and a pixel sensor (PS). The PS
modules with their granularity allow for high resolution, while keeping the
occupancy low. The 2S modules on the other hand limit the bandwidth and power
consumption and subsequently the material needed in the tracking volume. The
different modules are built around different kind of ASICs that are used to read
out the silicon sensor information and form the local track segments. In the PS
module there are two different types of ASICs. The MPA is bump-bonded onto a
pixelated layer, while the SSA is wire-bonded to connect the strip layers. The
2S modules use the CBC ASIC.

The first prototype assembly for the MPA ASIC covers only a fraction of the
sensor size planned for the PS modules and was named MPA-Light, and its module
assembly MaPSA-Light. I set up a laser system at Fermilab, that triggered the
data taking and measured the timewalk for this device for the first time with a
sensor attached.

For the CBC ASIC, I characterized all three full-sized prototype modules that
were available at the time. The resulting measurements of the mean noise were
released in the Technical Design Report of the CMS Phase-2 tracking system. For
this measurements I have set up a readout system that is based on a GLIB readout
card. In the bump-bonding lab at CERN, I have set up the same readout system, in
order to test and measure the newly produced modules during and right after
production. Since this was the first time the modules have been thoroughly
characterized, I discovered several flaws in the firmware, software and module
design, all of them have been improved by now.

The same read-out setup has also been used to test the flexible circuits that
are used to connect the two silicon strip layers in a 2S module. These hybrids
have been commercially manufactured and I have done electrical tests to assure
the quality of the prototypes. I automated the testing procedure completely by
software, significantly speeding up the process. Also the generation of a test
report was automated. The hybrid circuits that passed the quality assurance
tests as well as the visual inspection, were shipped to various CMS member
institutions for the construction of prototype modules. Thanks to the
automatization of the quality assurance test process, the results can easily be
reproduced by the institutes.

At Fermilab I have set up a crate based read out, that was first used to read
out a smaller prototype of the CBC chip and later to test version 3 of the CBC
chip for the first time in a test beam. The crate based read out used a uTCA
card developed at CERN called FC7. Since the FC7 based read-out offers more
flexibility in terms of reading out module prototypes, I have set up the same
FC7 based read out at CERN.

I have contributed to several test beams by setting up the systems and allow for
efficient data taking. The measurements obtained during these test beams are to
be published in two separate publications, both are in preparation right now.

\vspace{15pt}
\Large{}
\textit{Data acquisition with \textit{otsdaq}}
\normalsize{}

I also contributed to the development of \textit{otsdaq}, which is an
off-the-shell DAQ solution for high energy physics experiments. I adopted the
software to be used for the Phase-2 Outer Tracker modules and implemented the
existing Phase-2 Outer Tracker software framework into \textit{otsdaq}. I have
introduced the tool to the Phase-2 Outer Tracker community by giving
presentations and a hands-on session at the Phase-2 Outer Tracker Workshop.

Some of the highlights of \textit{otsdaq} are the finite state machine of the
detector, a graphical user interface that allows for simple and easy to learn
data taking, but also detector configuration, and live data quality monitoring,
as well as smaller subsystems, like an integrated e-log, messaging service or
macro maker.

\textit{otsdaq} has been commissioned in several test beams. At the last CMS
Phase-2 Outer Tracker test beam at Fermilab, it has been the standard tool for
data taking.

\vspace{15pt}
\Large{}
\textit{Test optical receivers for suitability with IBL detector}
\normalsize{}

The insertable b-layer (IBL) is an addition to the pixel detector from Run-1 in
ATLAS and was installed during the Long Shutdown 1. It is the layer closest to
the interaction point and allows for more precise vertex finding and track
separation. I have tested the commercial optical readout components to be used
off-detector. For this I have set up the laboratory at the University of Bern.
The acceptance criteria for the optical receivers with form factor SNAP 12 I
have defined to be reliability, sensitivity to the input light intensity and
frequency range. For testing the reliability, I have put them into a climate
chamber for accelerated aging and monitored them with a logic analyzer and self
programmed FPGA. I tested the receivers for efficiency as function of the input
sensitivity and the frequency and found that two of the three different vendors
matched all quality criteria. One of the two vendors was then chosen by the
community, also depending on external factors like availability and price. I
have presented my measurements in several IBL General Meetings and my results
have been an important decision factor in choosing the right product. The
receivers were later installed in the cavern and are still in operation. I also
replaced optical transmitters that ceased to function during operation in Run-1.

\end{statement}
\fi

\end{document}
